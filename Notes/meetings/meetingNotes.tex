\documentclass{article}
\usepackage[utf8]{inputenc}

\title{Notes on meetings ASC 2021}
\author{Maximilian Williams }
\date{August 2021}

\begin{document}

\maketitle

\section*{What is this?}
Quick notes on what I remember from each meeting. 

\section*{10th of September 2021}
\subsection*{General Notes}
\begin{enumerate}
	\item $p = p_{hydrostatic}$ will not work. It does not account for dynamic pressure terms. We need a more complex method.
	\item Read some engineering texts on how they typically deal with these pressure terms
	\item Look at the stream function approach in 2D. Should be pretty simple to code this up. Look at the affect of the inner boundry on results of the code.
	\item Read the two papers on the lattice-boltzman method. Might look at solving the full 3D problem like this
\end{enumerate}


\subsection*{Main goals for this week}
\begin{enumerate}
	\item A broad understanding of how other people solve these problems
	\item A simple 2D simulation
	\item An approximate path for solving this problem
\end{enumerate}

\section*{23rd of september 2021}

\subsection*{General Notes}

\begin{enumerate}
	\item We discussed convection in a box
	\item Code I have now looks approximately correct
	\item Should look at adding ghost points. These ghost points will allow for a method of images style of temperature boundry condtion. It should also allow for faster processing allowing for simple vector operations.
	\item More testing of this simple code should be done
	\item Go test if diffusion is working
	\item Use the "cone test" to see if advection is working. Take a parcel of fluid and transport it around, see if it deforms by taking it around a circle.
	\item I need to think a bit more about the above two
	\item Next we looked at modelling this on a disk
	\item Its probably not the case that the streamfunction method doesnt give an eligant solution in polar coordinates so keep trying
	\item Talking about not being able to access a lot of LBM papers
\end{enumerate}

\newpage











\end{document}

