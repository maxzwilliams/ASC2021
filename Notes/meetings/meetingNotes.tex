\documentclass{article}
\usepackage[utf8]{inputenc}

\title{Notes on meetings ASC 2021}
\author{Maximilian Williams }
\date{August 2021}

\begin{document}

\maketitle

\section*{What is this?}
Quick notes on what I remember from each meeting. 

\section*{10th of September 2021}
\subsection*{General Notes}
\begin{enumerate}
	\item $p = p_{hydrostatic}$ will not work. It does not account for dynamic pressure terms. We need a more complex method.
	\item Read some engineering texts on how they typically deal with these pressure terms
	\item Look at the stream function approach in 2D. Should be pretty simple to code this up. Look at the affect of the inner boundry on results of the code.
	\item Read the two papers on the lattice-boltzman method. Might look at solving the full 3D problem like this
\end{enumerate}


\subsection*{Main goals for this week}
\begin{enumerate}
	\item A broad understanding of how other people solve these problems
	\item A simple 2D simulation
	\item An approximate path for solving this problem
\end{enumerate}

\section*{23rd of september 2021}

\subsection*{General Notes}

\begin{enumerate}
	\item We discussed convection in a box
	\item Code I have now looks approximately correct
	\item Should look at adding ghost points. These ghost points will allow for a method of images style of temperature boundry condtion. It should also allow for faster processing allowing for simple vector operations.
	\item More testing of this simple code should be done
	\item Go test if diffusion is working
	\item Use the "cone test" to see if advection is working. Take a parcel of fluid and transport it around, see if it deforms by taking it around a circle.
	\item I need to think a bit more about the above two
	\item Next we looked at modelling this on a disk
	\item Its probably not the case that the streamfunction method doesnt give an eligant solution in polar coordinates so keep trying
	\item Talking about not being able to access a lot of LBM papers
\end{enumerate}

\section*{1st of October 2021}

\subsection*{General Notes}

\begin{enumerate}

	\item Clean up the plots so you can actually read them and use a polar projection for the disk code
	\item Add velocity arrows and contours to the plots
	\item Look closely at the boundries, make sure that I have used the correct boundary conditions
	\item My scheme is not stable. This is because of the advection term in the the temperature equation.
	\item Look in Numerical Recepies/in the notes for some solutions to this
	\item Semi-lagrange Crank-Nickleson scheme is a scheme that I could try, its unconditionally stable and accurate
	\item REALLY IMPORTANT: I need to find a scheme that is stable and accurate, without it the base programs will not work
	\item Once I get the a stable scheme I should work on LBM code (if I have time)
	\item I should also do the checks that my code is working from last week that I didnt do

\end{enumerate}




\section*{8th of October 2021}
\begin{enumerate}
	\item Godunov scheme looks about right
	\item Look carefully at the conditions for stability and accuracy, make sure they are consistant with the inner core
	\item I should probably write everything in dimensionless form (both code and report)
	\item Test the advection of my current codes. Set diffusion to 0, manually set the flow to some constant everywhere 
	and see how my initial condition is advected about. Do diagonal, up and down and use the periodic boundry conditions 
	to go back to the beginning position.
	\item Things to look for in this test are where the centroid of my shape goes, does the centroid get displaced about the correct amount (I might lag) and how deformed is my initial shape once advected. It should be pretty deformed.
	\item If all that works, lets get onto the real stuff. 
	\item Model internal heating and see if the size of that inner boundry matters. How does do the polar and cartesian compare? Is the critical Reyleigh number about right in these simulations? 
	\item If all that goes to plan, then LBM begins.
\end{enumerate}

\newpage

\section*{18th of October 2021}
\begin{enumerate}
	\item LB code looks good
	\item Advection tests look alright
	\item Look at how the hole in the middle impacts the simulation
	\item Does it matter how large the hole in the middle is? Does it change if we get convection or not? How do the patterns developed by convection change?
	\item I also need to consider the inner boundry condition on the streamfunction-vorticity codes, do I have a constant heat flux, no heat flux? How does
	this change the results?
	\item The main result we want to get however is still about convection in the Earths inner core. 
	\item Think about how I will define Ra. (Watch zoom video again)
	\item Then I reall want to answer the main question. Give the core some uniform internal heating and make the outer boundary cooled. Then look at what conditions 
	are needed to get convection here. Use a big Pr and see what Ra I will get.
	\item Once I know where convection will occure, lets look at the types of plumes that develop, what their length scale etc.
\end{enumerate}

\newpage


















\end{document}

