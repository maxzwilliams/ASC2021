\documentclass{article}
\usepackage[utf8]{inputenc}
\usepackage{graphicx}
\usepackage[a4paper, total={8in, 10.5in}]{geometry}
\usepackage{rotating}
\usepackage{afterpage}
\usepackage{url}




\title{Convection In A Box}
\author{Maximilian Williams}
\date{June 2021}

\begin{document}

\maketitle

\section*{What is this?}
A document where I can write down my working for modelling convection in a box as paper isn't the best at conveying ideas remotely. This is in no way polished document.

\section*{The Problem}
We seek to model 2D convection for a slow moving fluid. We will format this problem using streamfunctions with the aim of solving it numerically. First, the solution in a box using cartesian cooridnates is deicussed. The problem is then formatted in polar coordinates and solved on a disk.
\section*{The bousinesque Approximation}
In the bousinique approximation we assume density flucations are small. This leads us to considering only density flucations when they are multiplies by a gravity term.
\newline
We begin with the NSE, equation \ref{NSE}. We approximate the density by
\begin{equation}
	\rho = \rho_0 + \rho',
	\label{b density}
\end{equation}
where $\rho_0$ is a constant reference density and $\rho' << \rho_0$ is a pertibuation that depends on space. We similary split pressure up by equation \ref{b pressure}. 
\begin{equation}
	p = p_0 + p'
	\label{b pressure}
\end{equation}
Applying equation \ref{b density} and \ref{b pressure}, the Navier-Stokes equation reads:
\begin{equation}
	(\rho_0 + \rho') \frac{\partial \vec{u}}{\partial t} + (\rho_0 + \rho') (\vec{u} \cdot \nabla) \vec{u} = (\rho_0 + \rho') \vec{g} - \nabla (p_0 + p') + \mu \nabla^2 \vec{u}
	\label{b equation 1}
\end{equation}
By assuming $\vec{u}$ is also first order, equation \ref{b equation 1} to zeroth order produces:
\begin{equation}
	\rho_0 \vec{g} = \nabla p_0,
\end{equation}
and so to first order:

\begin{equation}
	\frac{\partial \vec{u}}{\partial t} + (\vec{u} \cdot \nabla) \vec{u} =  \frac{\rho'}{\rho_0}\vec{g} - \frac{\nabla p'}{\rho_0} + \nu \nabla^2 \vec{u},
	\label{b equation 2}
\end{equation}
where $\mu = \rho \nu$. 
\newline
And thats the bousinesque equation.

\section*{Equations}
The navier stokes equation (equation \ref{NSE}) describes the conservation of momentum for an incompressable fluid.
\begin{equation}
	\rho \frac{\partial \vec{u}}{\partial t} + \rho (\vec{u} \cdot \nabla) \vec{u} = \rho \vec{g} - \ \rho {\nabla p} + \mu \nabla^2 \vec{u}
	\label{NSE}
\end{equation}
The density $\rho$ is assumed to be a function of temperature $T$ and governed by equation \ref{equation of state}

\begin{equation}
	\rho = \rho_0 (1- \alpha (T - T_0)),
	\label{equation of state}
\end{equation}
where $\alpha$ is a coeffeicnet of thermal expansion and $\rho_0$ and $T_0$ are refernece densities and temperatures. The temperature is non constant, and modelled using equation \ref{advection-diffusion}
\begin{equation}
	\frac{\partial T}{\partial t} + (\vec{u} \cdot \nabla) T = \kappa \nabla^2 T + \frac{Q}{C_p},
	\label{ade}
\end{equation}
where $\kappa$ is the diffusion constant, $C_p$ the specific heat capacity per volume and $Q$ a heat source.
\newline
We apply the creeping flow approximation first, giving:
\begin{equation}
	\rho \frac{\partial \vec{u}}{\partial t} = \rho \vec{g} - {\nabla p} + \mu \nabla^2 \vec{u}
	\label{NSE slow}
\end{equation}
Next we apply the bousinesque approximation giving:
\begin{equation}
	\frac{\partial \vec{u}}{\partial t} = \frac{\rho'}{\rho} \vec{g} -   \frac{\nabla p'}{\rho_0} + \nu \nabla^2 \vec{u}
	\label{NSE slow + b}
\end{equation}
We now aim to numerically solve equations \ref{NSE slow + b}, \ref{ade} and \ref{equation of state} using a finite difference scheme.

\section*{Streamfunction-vorticity formulation in cartesian coorindates}
We consider the problem in 2D cartesian co-ordinates (x,y) with $\vec{u}=(u,v)$, where $u$ and $v$ are the $x$ and $v$ velocities of the fluid respectively. We define a streamfunction $\psi$ by equation \ref{streamfunction equation}
\begin{equation}
	u = - \frac{\partial \psi}{\partial y}, v = \frac{\partial \psi}{\partial x}.
\end{equation}
We also define the vorticity in the $x-y$ plane $\omega$ by:
\begin{equation}
	\omega = \nabla \times \vec{u} = \frac{\partial v}{\partial x} - \frac{\partial u}{\partial y}.
	\label{vorticity}
\end{equation}
By these definitions of the streamfunction and vorticity we get the incompressability condition (equation \ref{incomp sf}) and the vorticity-streamfunction relation, equation \ref{vt to st}:

\begin{equation}
	0=\frac{\partial^2 \psi}{\partial x \partial y} - \frac{\partial^2 \psi}{\partial x \partial y} = \frac{\partial u}{\partial x} + \frac{\partial v}{\partial y} = \nabla \cdot \vec{u}
	\label{incomp sf}
\end{equation}

\begin{equation}
	\omega = \frac{\partial^2 \psi}{\partial x^2} + \frac{\partial^2 \psi}{\partial y^2}.
	\label{vt to st}
\end{equation}
Using the streamfunction we can immediately rewrite equation \ref{ade} as equation \ref{ade sf}.
\begin{equation}
	\frac{\partial T}{\partial t} -\frac{\partial \psi}{\partial y} \frac{\partial T}{\partial x} + \frac{\partial \psi}{\partial x} \frac{\partial T}{\partial y} = \kappa \nabla^2 T + \frac{Q}{C_p}
	\label{ade sf}
\end{equation}
The streamfunction-vorticity equation for the Navier Stokes equation is a little more involved. We first split equation \ref{NSE slow + b} into its components:
\begin{equation}
	\frac{\partial u}{\partial t} = \frac{\rho'}{\rho_0} g_x -\frac{1}{\rho_0} \frac{\partial p'}{\partial x} + \nu (\frac{\partial^2 u}{\partial x^2} + \frac{\partial^2 u}{\partial x^2})
	\label{x comp}
\end{equation}
and,
\begin{equation}
		\frac{\partial v}{\partial t} = \frac{\rho'}{\rho_0} g_y -\frac{1}{\rho_0} \frac{\partial p'}{\partial y} + \nu (\frac{\partial^2 v}{\partial x^2} + \frac{\partial^2 v}{\partial x^2}).
		\label{y comp}
\end{equation}
Taking $\frac{\partial}{\partial x}$ of equation \ref{y comp} and subtracting $\frac{\partial }{\partial y}$ of equation \ref{x comp} we rearange to get equation \ref{NSE sf+v}:
\begin{equation}
	\frac{\partial w}{\partial t} = \frac{g_y}{\rho_0} \frac{\partial \rho'}{\partial x} - \frac{g_x}{\rho_0} \frac{\partial \rho'}{\partial y} + \nu \nabla^2 \omega
	\label{NSE sf+v}
\end{equation}
Equations \ref{NSE sf+v} and \ref{ade sf} complete our streamfunction-vortivity description.

\section*{Boundry conditions}
There are two classes of boundry conditions we must adress, fluid related boundy conditions and temperature related boundry conditions. We first consider the fluid related boundry conditions. We take the four boundries as stationary impermiable walls. (I cant draw latex graphics yet so I dont have a figure for this). We have four walls at $x=0,x=L,y=0,y=L$. At each wall, the normal fluid velocity is zero. For example, at the $y=0$ and $y=L$ walls, we have:

\begin{equation}
	u = \frac{\partial \psi}{\partial y} = 0
\end{equation}
and thus,
\begin{equation}
	\psi = constant.
	\label{streamfunction bc}
\end{equation}
This is true for all four walls, which are connected. So on the walls, $\psi = constant$.
From equation \ref{vorticity} on the left and right walls:
\begin{equation}
	\frac{\partial^2 \psi}{\partial x^2} + \frac{\partial^2 \psi}{\partial y^2} = \frac{\partial^2 \psi}{\partial y^2}= \omega_{wall}
	\label{vorticity left-right}
\end{equation}
and similary on the top and bottom wall:
\begin{equation}
	\frac{\partial^2 \psi}{\partial x^2} + \frac{\partial^2 \psi}{\partial y^2} = \frac{\partial^2 \psi}{\partial x^2}= \omega_{wall}.
	\label{vorticity top-bottom}
\end{equation}
Equations \ref{streamfunction bc}, \ref{vorticity left-right} and \ref{vorticity top-bottom} compose the boundry conditions used here.


\section*{Streamfunction-vorticity formulation in polar coorindates}
We consider polar coorindates $r$ and $\theta$ with $r\in \{ 0, ..., R\}$ and $\theta \in \{ 0, ... , 2\pi\}$. We discritize our domain $\mathcal{D}$ with grids points  $(r_i,\theta_j)$ with $r_i = i \Delta r, \theta_j = j \Delta \theta$. We define a velocity vector $\vec{u} = (u, v)$ with $u$ pointing radially and $v$ pointing azimuthally. A streamfunction $\psi$ is used to define $u$ and $v$ as shown:
\begin{equation}
	u = \frac{1}{r} \frac{\partial \psi}{\partial \theta}, v = - \frac{\partial \psi}{\partial r}.
\end{equation}
In defining $u$ and $v$ this way, the incompressability condition (equation \ref{polar compress}) is satisfied:
\begin{equation}
	\nabla \cdot \vec{u} = 0,
	\label{polar compress}
\end{equation}
where $\nabla$ is the del operator in cylindrical coordinates with any terms terms out of the $r-\theta$ plane supressed. The voriticty, $\omega$ is defined as:
\begin{equation}
	\omega = (\nabla \times \vec{u}) \cdot \hat{z} = - \nabla^2 \psi,
	\label{omega polar}
\end{equation}
where $\hat{z}$ points out of the $r-\theta$ plane.
\newline
The simplified navier stokes equation (equation \ref{NSE slow + bs}) can be written in polar co-ordinates as:
\begin{equation}
	\frac{\partial u}{\partial t} = -\frac{1}{\rho_0} \frac{\partial p'}{\partial r} + \frac{\rho'}{\rho_0} g_r + \nu (  \nabla^2 u - \frac{u}{r^2} - \frac{2}{r^2} \frac{\partial v}{\partial \theta}  )
	\label{u polar}
\end{equation}
\begin{equation}
	\frac{\partial v}{\partial t} = -\frac{1}{r \rho_0} \frac{\partial p'}{\partial \theta} + \frac{\rho'}{\rho_0} g_{\theta} + \nu ( \nabla^2 v - \frac{v}{r^2} + \frac{2}{r^2} \frac{\partial u}{\partial \theta})
	\label{v polar}
\end{equation}
Multiplying equation \ref{v polar} by $r$ and cross-differentationing $\frac{\partial}{\partial r}$ r (\ref{v polar}) $- \frac{\partial}{\partial \theta}$ (\ref{u polar}) we get:
\begin{equation}
	\frac{\partial}{\partial t} (r \frac{\partial v}{\partial r} - \frac{\partial u}{\partial \theta}) = -\frac{g_r}{\rho_0} \frac{\partial \rho'}{\partial \theta} + \nu( (\frac{\partial}{\partial r} r(\nabla^2 u - \frac{u}{r^2} - \frac{2}{r^2} \frac{\partial v}{\partial \theta})) - (\frac{\partial}{\partial \theta} (\nabla^2 v - \frac{v}{r^2} + \frac{2}{r^2} \frac{\partial u}{\partial \theta}) ) ),
	\label{polar 1}
\end{equation}
where we have assumed that $\theta$ and $r$ derivatives commute and $g_{\theta}=0$. Using the definition of $\omega$ in equation \ref{omega polar} we write equation \ref{polar 1} as:
\begin{equation}
	\frac{\partial \omega}{\partial t} = - \frac{g_r}{\rho_0 r} \frac{\partial \rho'}{\partial \theta} +\nu \nabla^2 \omega
	\label{polar vorticity}
\end{equation}















 


















\end{document}
