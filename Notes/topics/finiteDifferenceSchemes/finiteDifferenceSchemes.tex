\documentclass{article}
\usepackage[utf8]{inputenc}
\usepackage{graphicx}
\usepackage[a4paper, total={8in, 10.5in}]{geometry}
\usepackage{rotating}
\usepackage{afterpage}
\usepackage{url}




\title{Finite Difference Schemes}
\author{Maximilian Williams}
\date{September 2021}

\begin{document}

\maketitle

\section*{Finite Difference Schemes}
There are four typical metrics used to evalaute a finite difference scheme. Accuracy, Consistency, Stability and Convergence.

\subsection*{Accuracy}
The order of terms neglected in the finite difference approximation is the accuracy

\subsection*{Consistency}
A finite difference approximation is considered consistent if by refining the mesh both temporally and spatially the truncation error goes to zero.

\subsection*{Stability}
A finite difference whose truncation errors grow unbounded as the solution evolves is considered unstable. Else, the finite difference is stable.

\subsection*{Convergence}
We will not worry too much about this, but its the rate at which mesh refinement decreases the truncation error.



\end{document}
