\documentclass{article}
\usepackage[utf8]{inputenc}
\usepackage{graphicx}
\usepackage[a4paper, total={8in, 10.5in}]{geometry}
\usepackage{rotating}
\usepackage{afterpage}
\usepackage{url}




\title{Numerical Analysis of Convection in the Inner Core (DRAFT)}
\author{Maximilian Williams}
\date{September 2021}

\begin{document}

\maketitle

\begin{abstract}
	Convection in the Earth's inner core has been a contentious topic in geoscience. Recently, it has been proposed through siesmic observations that Earths inner core is convecting. Here we numerically model convection in the Inner core, using the streamfunction-voriticty formulation in 2 dimensions and a three dimensional lattice boltzman approach. We find ...
	
 
\end{abstract}


\section*{Introduction}
Plan:
I want to also talk about why this is actually important, why is this something that is worth studying
\newline
Here I want to introduce the physics of what I want to talk about. I want to introduce 
the basic equations that I will use.
\newline
\noindent Throughout analysis we describe the fluid in the Eularian frame under a graviational acceleration $\vec{g}$ which may vary in space. We give each location in the fluid a velocity $\vec{u}$ and density $\rho$ that vary in space $\vec{x}$ and time $t$. We assume the fluids viscosity $\mu$, thermal diffusivity $\kappa$ and specific heat capacity $C_p$ are all constants. By conserving fluid momentum, we produce the navier stokes equation:
\begin{equation}
	\rho \frac{D \vec{u}}{D t} = \rho \vec{g} - \nabla p + \mu \nabla^2 \vec{u},
	\label{NSE}
\end{equation}
where $\frac{D}{D t} = \frac{\partial }{\partial t} + (\vec{u} \cdot \nabla)$ is the material derivative, $p$ the pressure of the fluid and $\nabla$ the del operator. The dynamics of fluid temperature $T$ are described the inhomoenous advection diffusion equation:
\begin{equation}
	\frac{D T}{D t} = \kappa \nabla^2 T + \frac{Q}{C_v \rho},
	\label{adeT}
\end{equation}
where $Q$ is the heat generated per unit volume per unit time, $C_v$ the specific heat at constant volume and $\rho$ the fluid density. The density of the fluid $\rho$ is assumed to vary linearly in temeprature according to the equation of state:
\begin{equation}
	\rho = \rho_0 (1- \alpha(T - T_0)),
	\label{equation of state}
\end{equation}
where $\alpha$ is the volumetric expansion coeffecient and $\rho_0$ the density at a reference temperature $T_0$.
Through siesmic imaging, variations in inner core density are $<< 1 \%$. We also assume that the inner cores evolution occures over geologic timescales, and as such take $\vec{u}$ to be first order. These assumptions allow us to make the slow flow boussinesq approximation to equation \ref{NSE}:
\begin{equation}
	\frac{\partial \vec{u}}{\partial t} = \frac{\rho'}{\rho} \vec{g} -   \frac{\nabla p'}{\rho_0} + \nu \nabla^2 \vec{u},
	\label{NSE slow + boussinesq}
\end{equation}
where $\rho'=-\alpha(T - T_0)$, $\nu$ the kinomatic viscosity $\nu = \frac{\mu}{\rho_0}$ and $p'$ a first order pertibation to the background pressure $p_0$.
\newline
The Earths











\subsection*{Numerical Methods}
Here I want to introduce my two numerical schemes

\subsubsection*{Streamfunction-Vorticity formulation}
The streamfunction-vorticity formulation is a popular method for analytical and simple numerical analysis of incompressable fluids in two dimensions. Its key advantage is the elimination of all pressure terms, which would otherwise need to be iteratively accounted for or given in a constituative equation. We define the vorticity in the plane $\omega$ by:
\begin{equation}
	\omega = (\nabla \times \vec{u})_z,
	\label{omega}
\end{equation}
where the $z$ subscript denotes the component out of the page. We also define a streamfunction $\psi$ by:
\begin{equation}
	\omega = \nabla^2 \psi.
	\label{psi}
\end{equation}
Physically, the voriticty is the amount of spinning the fluid does about a point, while lines of constant streamfunction have the fluid flow perpendicular to them. Given a coordinate system, and a clever definition of $\vec{u}$ we can rewrite 
equations \ref{adeT} and \ref{NSE slow + boussinesq} in terms of $\omega$ and $\psi$ rather than $\vec{u}$. In cartesian coordinates $(x,y)$ we pick 
\begin{equation}
	u = -\frac{\partial \psi}{\partial y}, v = \frac{\partial \psi}{\partial x},
	\label{cartesian velocities}
\end{equation}
giving:
\begin{equation}
	\frac{\partial T}{\partial t} -\frac{\partial \psi}{\partial y} \frac{\partial T}{\partial x} + \frac{\partial \psi}{\partial x} \frac{\partial T}{\partial y} = \kappa \nabla^2 T + \frac{Q}{\rho_0 C_v}
	\label{adeT sfvt cartesian}
\end{equation}
\begin{equation}
	\frac{\partial w}{\partial t} = \frac{g_y}{\rho_0} \frac{\partial \rho'}{\partial x} + \nu \nabla^2 \omega
	\label{NSE slow + boussinesq sfvt cartesian}
\end{equation}
In polar coorindates (r, $\theta$) we pick:
\begin{equation}
	u = -\frac{1}{r} \frac{\partial \psi}{\partial \theta}, v = \frac{\partial \psi}{\partial r},
	\label{polar velocities}
\end{equation}
giving:
\begin{equation}
	\frac{\partial T}{\partial t} - \frac{1}{r} \frac{\partial \psi}{\partial \theta} \frac{\partial T}{\partial r} + \frac{1}{r} \frac{\partial \psi}{\partial r} \frac{\partial T}{\partial \theta} = \kappa \nabla^2 T + \frac{Q}{\rho_0 C_v}
	\label{adeT sfvt polar}
\end{equation}
and,
\begin{equation}
	\frac{\partial \omega}{\partial t} = - \frac{g_r}{\rho_0 r} \frac{\partial \rho'}{\partial \theta} +\nu \nabla^2 \omega.
	\label{NSE slow + boussinesq sfvt polar}
\end{equation}
A fully derivation of equations \ref{adeT sfvt cartesian}, \ref{NSE slow + boussinesq sfvt cartesian}, \ref{adeT sfvt polar}, \ref{NSE slow + boussinesq sfvt polar} are given in appendix.
Importantly, our definitions of $u$ and $v$ in equations \ref{cartesian velocities} and \ref{polar velocities} satisify equation \ref{omega} and \ref{psi}. Together, equations \ref{psi} \ref{adeT sfvt cartesian}, \ref{NSE slow + boussinesq sfvt cartesian}, \ref{adeT sfvt polar} and \ref{NSE slow + boussinesq sfvt polar} contain no unknown quantities like pressure $p$ and so can be solved via finite difference methods.

\subsubsection*{Solving the streamfunction-vorticity equations}
We first discritize our domain $\mathcal{D}$. In the cartesian case, we use $(x_i,y_j)=(i \Delta x, j \Delta y)$ with integers $i$ and $j$ satisfying $0\leq i < N_x$ $0 \leq j < N_y$. In the polar case, we use $(r_i, \theta_j)= (R_0 + i \Delta r, j 
\Delta \theta)$ again with  $0 \leq i < N_r$ and $0 \leq j < N_{\theta}$. We impose $\Delta \theta = \frac{2 \pi}{N_{\theta} - 1}$ for consistancy with $\theta$-periodic boundary conditions and an inner radius $R_0$ in polar coordinates to avoid 
singularities generated by $r=0$. We also discritize time $t$ by $t_n = n \Delta t$. For a function $f$, we use $f^n_{i,j}$ to mean $f$ evaluated at time $n$ at position $(x_i,y_j)$ in cartesian coorindates or $(r_i, \theta_j)$ in polar 
coordinates. 
\newline
To approximate derivatives we use a finite difference approach. All time derivatives are approximated by forward difference:
\begin{equation}
	\frac{\partial f}{\partial t} = \frac{f^{n+1} - f^{n}}{\Delta t}
	\label{forward time difference}
\end{equation}
Second order space derivatives are apprimxated by a central difference:
\begin{equation}
	\frac{\partial^2 f_{i,j}}{\partial {x_1}^2} = \frac{f_{i+1,j} - 2 f_{i,j} + f_{i-1,j}}{{\Delta x_1}^2},
\end{equation}
\begin{equation}
	\frac{\partial^2 f_{i,j}}{\partial {x_2}^2} = \frac{f_{i,j+1} - 2 f_{i,j} + f_{i,j-1}}{{\Delta x_2}^2},
\end{equation}
where $x_1$ is the first coordinate and $x_2$ is the second coordinate. For example, in cartesian coordinates $(x,y)$, we would have $x_1=x$ and $x_2=y$. 
For non advection terms, we approximate first order spatial derivaitves by:
\begin{equation}
	\frac{\partial f_{i,j}}{\partial x_1} = \frac{f_{i+1,j} - f_{i-1,j}}{2{\Delta x_1}},
\end{equation}
and
\begin{equation}
	\frac{\partial f_{i,j}}{\partial x_2} = \frac{f_{i,j+1} - f_{i,j+1}}{2{\Delta x_2}}.
\end{equation}
For advection terms, such as $a \frac{\partial f_{i,j}}{\partial x_1}$ we employ a first order godanov scheme:
\begin{equation}
	a \frac{\partial f_{i,j}}{\partial x_1} = \frac{1}{\Delta x} ( \mid a\mid (  \frac{1}{2} f_{i+1,j} - \frac{1}{2} f_{i-1,j}   ) - a ( \frac{1}{2} f_{i+1,j} -f_{i,j} - \frac{1}{2} f_{i-1,j} )).
\end{equation}
These scheme is always a downstream scheme regardless of the sign of the direction of the advecting field $a$.
\newline
Other more accurate, but substantually more complex methods for solving these equations, particularlly the advection equation exists such as the semi-lagrange crank-nicolson scheme.
\newline
To solve the streamfuntion-vorticity equations we assume a starting vorticity $\omega$ on our domain $\mathcal{D}$. We then apply the Jacobi method to solve equation \ref{psi} for $\psi$ on the interior of the domain which we call $\mathcal{D}'$. 
Using $\psi$ we update $T$ on $\mathcal{D}'$ using equation \ref{adeT sfvt cartesian} (or \ref{adeT sfvt polar} for polar). Finally, $\omega$ is updated on $\mathcal{D}'$ using 
equation \ref{NSE slow + boussinesq sfvt cartesian} (\ref{NSE slow + boussinesq sfvt polar} for polar). This process is repeated.

\subsubsection*{The Jacobi Method}
We cannot solve for $\psi$ explicitly in equation \ref{psi}. Instead we use an iterative method called the Jacobi method. In cartesian coorindates, equation \ref{psi} can be written:

\begin{equation}
	\omega = \frac{\partial^2 \psi}{\partial x^2} + \frac{\partial^2 \psi}{\partial y^2}.
\end{equation}
Using our finite differences this can be written as:
\begin{equation}
	\omega_{i,j} = \frac{\psi_{i+1,j} - 2 \psi_{i,j} + \psi_{i-1,j}  }{{\Delta x}^2} + \frac{\psi_{i,j+1} - 2 \psi_{i,j} + \psi_{i,j-1}  }{{\Delta y}^2}
	\label{psi disc}
\end{equation}
We can then write $\psi_{i,j}$ as:
\begin{equation}
	\psi_{i,j} = \frac{{\Delta x}^2 {\Delta y}^2  }{2*({\Delta x}^2  + {\Delta y}^2)} (\frac{\psi_{i+1,j} +\psi_{i-1,j} }{{\Delta x}^2} + \frac{\psi_{i,j+1} +\psi_{i,j-1}  }{{\Delta y}^2 }   - \omega_{i,j}).
	\label{psi from omega}
\end{equation}
We then use the result of $\psi_{i,j}$ back into equation \ref{psi from omega} to solve for $\psi_{i,j}$ iterately as shown in equation \ref{psi iterative}:
\begin{equation}
	\psi_{i,j}^{(k+1)} = \frac{{\Delta x}^2 {\Delta y}^2  }{2*({\Delta x}^2  + {\Delta y}^2)} (\frac{\psi_{i+1,j}^{(k)} +\psi_{i-1,j}^{(k)}  }{{\Delta x}^2} + \frac{\psi_{i,j+1}^{(k)} +\psi_{i,j-1}^{(k)}  }{{\Delta y}^2 }   - \omega_{i,j}).
	\label{psi iterative}
\end{equation}
Where the superscript $(k)$ means the result of the $k^{th}$ iteration of the above equation and this operation us applied to all points in $\mathcal{D}'$.
 We terminate this iterative method once the error $\sum_{(i,j) \in \mathcal{D'}} \mid \psi_{i,j}^{(k+1)} - \psi_{i,j}^{(k)} \mid$ gets suffeciently small. A similary method is applied to the polar coordinate case. 

\subsubsection*{Solution Stability and Accuracy}
%% Here I want to look at the stabiliy of this method
Here I will have some very simple stability analysis
for the advection equation and the $\omega$ equation.


\subsubsection*{Lattice Boltzman Method}
I want to introduce the lattice boltzman method, go over its derivation, talk about why this is unique 
in the world of computational fluid dynamics. I pretty much just want to explain what it is and how I used it.





\section*{Advection and Heat Conservation}
We next tested the accuracy of the advection schemes. Thermal diffusivity ($\kappa$) and thermal expansion ($\alpha$) were set to zero. In the first test, the streamfunction was set so that the velocity field was purely horizontal and azimuthal in the cartesian and polar cases respectively. 






To determine how accurate the advection schemes for the polar and cartesian vortivity-streamfunction programs, 


\section*{Results}

\section*{}





\end{document}
