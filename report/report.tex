\documentclass{article}
\usepackage[utf8]{inputenc}
\usepackage{graphicx}
\usepackage[a4paper, total={8in, 10.5in}]{geometry}
\usepackage{rotating}
\usepackage{afterpage}
\usepackage{url}

\usepackage{amsmath}




\title{Numerical Analysis of Convection in the Inner Core (DRAFT)}
\author{Maximilian Williams}
\date{September 2021}

\begin{document}

\maketitle

\begin{abstract}
	Convection in the Earth's inner core has been a contentious topic in geoscience. Recently, it has been proposed through siesmic observations that Earths inner core is convecting. Here we numerically model convection in the Inner core, using the streamfunction-vorticity formulation and lattice boltzman method in 2-dimensions. 
	
 
\end{abstract}

\section*{Introduction}
Plan:
I want to also talk about why this is actually important, why is this something that is worth studying
\newline
Here I want to introduce the physics of what I want to talk about. I want to introduce 
the basic equations that I will use.
\newline






\subsubsection*{Governing Equations}
{\it{In this section I introduce the physics of the problem; the governing equations that we wish to numerically solve.}}
\vspace{0.3cm}
\newline
\noindent Throughout analysis we describe the fluid in the Eularian frame under a graviational acceleration $\vec{g}$ which may vary in space. We give each location in the fluid a velocity $\vec{u}$ and density $\rho$ that vary in space $\vec{x}$ and time $t$. We assume the fluids viscosity $\mu$, thermal diffusivity $\kappa$ and specific heat capacity $C_p$ are all constants. By conserving fluid momentum, we produce the navier stokes equation:
\begin{equation}
	\rho \frac{D \vec{u}}{D t} = \rho \vec{g} - \nabla p + \mu \nabla^2 \vec{u},
	\label{NSE}
\end{equation}
where $\frac{D}{D t} = \frac{\partial }{\partial t} + (\vec{u} \cdot \nabla)$ is the material derivative, $p$ the pressure of the fluid and $\nabla$ the del operator. The dynamics of fluid temperature $T$ are described the inhomoenous advection diffusion equation:
\begin{equation}
	\frac{D T}{D t} = \kappa \nabla^2 T + \frac{Q}{C_v \rho},
	\label{adeT}
\end{equation}
where $Q$ is the heat generated per unit volume per unit time, $C_v$ the specific heat at constant volume and $\rho$ the fluid density. The density of the fluid $\rho$ is assumed to vary linearly in temeprature according to the equation of state:
\begin{equation}
	\rho = \rho_0 (1- \alpha(T - T_0)),
	\label{equation of state}
\end{equation}
where $\alpha$ is the volumetric expansion coeffecient and $\rho_0$ the density at a reference temperature $T_0$.
Through siesmic imaging, variations in inner core density are $<< 1 \%$. We also assume that the inner cores evolution occures over geologic timescales, and as such take $\vec{u}$ to be first order. These assumptions allow us to make the slow flow boussinesq approximation to equation \ref{NSE}:
\begin{equation}
	\frac{\partial \vec{u}}{\partial t} = \frac{\rho'}{\rho} \vec{g} -   \frac{\nabla p'}{\rho_0} + \nu \nabla^2 \vec{u},
	\label{NSE slow + boussinesq}
\end{equation}
where $\rho'=-\alpha(T - T_0)$, $\nu$ the kinomatic viscosity $\nu = \frac{\mu}{\rho_0}$ and $p'$ a first order pertibation to the background pressure $p_0$.
\newline
The Earths




\subsection*{Numerical Methods}

{\it{Two numerical methods are introduced for solving the thermal convection problem, the Lattice Boltzman Method and the streamfunction-vorticity formulation. }}

\subsubsection*{Streamfunction-Vorticity formulation}
{\it{Here I introduce the streamfunction-vorticity method for use in 2 dimensions and use it to eliminate pressure terms in \ref{NSE slow + boussinesq} and \ref{adeT} in cartesian and polar geometries giving a set of numerically solvable equations.}}
\vspace{0.3cm}
\newline
\noindent The streamfunction-vorticity formulation is a popular method for analytical and simple numerical analysis of incompressable fluids in two dimensions. Its main advantage is its elimination of all pressure terms, which would typically require iterative techniques to solve, as is used in the SIMPLE algorithm. However, the streamfunction-vorticity method is limited to 2-dimensional or 3-dimensional symmetric flows and so has limited applicability.

We define two scalar quantities, the vorticity $\omega$ and the streamfunction $\psi$. The vorticity $\omega$ is given by:
\begin{equation}
	\omega = (\nabla \times \vec{u})_z,
	\label{omega}
\end{equation}
where the $z$ subscript denotes the component out of the plane. We also define a streamfunction $\psi$ by:
\begin{equation}
	\omega = - \nabla^2 \psi.
	\label{psi}
\end{equation}
Given a coordinate system, and a clever definition of $\vec{u}$ we can rewrite equations \ref{adeT} and \ref{NSE slow + boussinesq} in terms of $\omega$ and $\psi$ rather than $\vec{u}$ and $p$. In cartesian coordinates $(x,y)$ we pick 
\begin{equation}
	u = \frac{\partial \psi}{\partial y}, v = -\frac{\partial \psi}{\partial x},
	\label{cartesian velocities}
\end{equation}
allowing us to write equations \ref{adeT} and \ref{NSE slow + boussinesq} as:
\begin{equation}
	\frac{\partial T}{\partial t} + \frac{\partial \psi}{\partial y} \frac{\partial T}{\partial x} - \frac{\partial \psi}{\partial x} \frac{\partial T}{\partial y} = \kappa \nabla^2 T + \frac{Q}{\rho_0 C_v}
	\label{adeT sfvt cartesian}
\end{equation}
\begin{equation}
	\frac{\partial w}{\partial t} = -\frac{g_y}{\rho_0} \frac{\partial \rho'}{\partial x} + \nu \nabla^2 \omega
	\label{NSE slow + boussinesq sfvt cartesian}
\end{equation}
Similary, in polar coordinates (r, $\theta$) we pick:
\begin{equation}
	u = \frac{1}{r} \frac{\partial \psi}{\partial \theta}, v = -\frac{\partial \psi}{\partial r},
	\label{polar velocities}
\end{equation}
giving:
\begin{equation}
	\frac{\partial T}{\partial t} + \frac{1}{r} \frac{\partial \psi}{\partial \theta} \frac{\partial T}{\partial r} - \frac{1}{r} \frac{\partial \psi}{\partial r} \frac{\partial T}{\partial \theta} = \kappa \nabla^2 T + \frac{Q}{\rho_0 C_v}
	\label{adeT sfvt polar}
\end{equation}
and,
\begin{equation}
	\frac{\partial \omega}{\partial t} = - \frac{g_r}{\rho_0 r} \frac{\partial \rho'}{\partial \theta} +\nu \nabla^2 \omega.
	\label{NSE slow + boussinesq sfvt polar}
\end{equation}
A full derivation of equations \ref{adeT sfvt cartesian}, \ref{NSE slow + boussinesq sfvt cartesian}, \ref{adeT sfvt polar}, \ref{NSE slow + boussinesq sfvt polar} are given in appendix.
Importantly, our definitions of $u$ and $v$ in equations \ref{cartesian velocities} and \ref{polar velocities} satisify equation \ref{omega} and \ref{psi}. Equations \ref{psi}, \ref{adeT sfvt cartesian}, \ref{NSE slow + boussinesq sfvt cartesian} for the cartesian case and \ref{psi}, \ref{adeT sfvt polar}, \ref{NSE slow + boussinesq sfvt polar} for the polar case can be directly solved.

\subsubsection*{Solving the streamfunction-vorticity equations}
{\it{The finite difference method used for solving the streamfunction-vorticity-formulated governing equations is shown}}
\vspace{0.3cm}
\newline
\noindent We first discritize our domain $\mathcal{D}$. In the cartesian case, we use $(x_i,y_j)=(i \Delta x, j \Delta y)
$ with integers $i$ and $j$ satisfying $0\leq i < N_x$ $0 \leq j < N_y$. In the polar case, we use $(r_i, \theta_j)= (R_0 
+ i \Delta r, j 
\Delta \theta)$ again with  $0 \leq i < N_r$ and $0 \leq j < N_{\theta}$. We impose $\Delta \theta = \frac{2 \pi}
{N_{\theta} - 1}$ for consistancy with $\theta$-periodic boundary conditions and an inner radius $R_0$ in polar 
coordinates to avoid 
singularities generated by $r=0$. We also discritize time $t$ by $t_n = n \Delta t$. For a function $f$, we use 
$f^n_{i,j}$ to mean $f$ evaluated at time $n$ at position $(x_i,y_j)$ in cartesian coorindates or $(r_i, \theta_j)$ in 
polar coordinates. 
\newline
To approximate derivatives we use a finite difference approach. All time derivatives are approximated by forward difference:
\begin{equation}
	\frac{\partial f}{\partial t} = \frac{f^{n+1} - f^{n}}{\Delta t}
	\label{forward time difference}
\end{equation}
Second order space derivatives are apprimxated by a central difference:
\begin{equation}
	\frac{\partial^2 f_{i,j}}{\partial {x_1}^2} = \frac{f_{i+1,j} - 2 f_{i,j} + f_{i-1,j}}{{\Delta x_1}^2},
\end{equation}
\begin{equation}
	\frac{\partial^2 f_{i,j}}{\partial {x_2}^2} = \frac{f_{i,j+1} - 2 f_{i,j} + f_{i,j-1}}{{\Delta x_2}^2},
\end{equation}
where $x_1$ is the first coordinate and $x_2$ is the second coordinate. For example, in cartesian coordinates $(x,y)$, we would have $x_1=x$ and $x_2=y$. 
For non advection terms, we approximate first order spatial derivaitves by:
\begin{equation}
	\frac{\partial f_{i,j}}{\partial x_1} = \frac{f_{i+1,j} - f_{i-1,j}}{2{\Delta x_1}},
\end{equation}
and
\begin{equation}
	\frac{\partial f_{i,j}}{\partial x_2} = \frac{f_{i,j+1} - f_{i,j+1}}{2{\Delta x_2}}.
\end{equation}
For advection terms, of the form $a \frac{\partial f_{i,j}}{\partial x_1}$ we employ a first order godanov scheme:
\begin{equation}
	a \frac{\partial f_{i,j}}{\partial x_1} = \frac{1}{\Delta x} ( \mid a\mid (  \frac{1}{2} f_{i+1,j} - \frac{1}{2} f_{i-1,j}   ) - a ( \frac{1}{2} f_{i+1,j} -f_{i,j} - \frac{1}{2} f_{i-1,j} )).
\end{equation}
This scheme is always downstream, regardless of the direction of the advecting field $a$.
\newline
Other more accurate, but substantually more complex methods for solving these equations, particularlly the advection equation exists such as the semi-lagrange crank-nicolson scheme.
\newline
To solve the streamfuntion-vorticity equations we assume a starting vorticity $\omega$ on our domain $\mathcal{D}$. We then apply the Jacobi method to solve equation \ref{psi} for $\psi$ on the interior of the domain which we call $\mathcal{D}'$. 
Using $\psi$ we update $T$ on $\mathcal{D}'$ using equation \ref{adeT sfvt cartesian} (or \ref{adeT sfvt polar} for polar). Finally, $\omega$ is updated on $\mathcal{D}'$ using 
equation \ref{NSE slow + boussinesq sfvt cartesian} (\ref{NSE slow + boussinesq sfvt polar} for polar). This process is repeated.

\subsubsection*{The Jacobi Method}
{\it{Here a basic numerical method for solving the Poisson equation, the Jacobi method is outlined.}}
We cannot solve for $\psi$ explicitly in equation \ref{psi}. Instead we use an iterative jacobi method. In cartesian coordinates, equation \ref{psi} is:
\begin{equation}
	\omega_{i,j} = \frac{\psi_{i+1,j} - 2 \psi_{i,j} + \psi_{i-1,j}  }{{\Delta x}^2} + \frac{\psi_{i,j+1} - 2 \psi_{i,j} + \psi_{i,j-1}  }{{\Delta y}^2}
	\label{psi disc}
\end{equation}
Rearanging for $\psi$
\begin{equation}
	\psi_{i,j} = \frac{{\Delta x}^2 {\Delta y}^2  }{2({\Delta x}^2  + {\Delta y}^2)} (\frac{\psi_{i+1,j} +\psi_{i-1,j} }{{\Delta x}^2} + \frac{\psi_{i,j+1} +\psi_{i,j-1}  }{{\Delta y}^2 }  + \omega_{i,j}).
	\label{psi from omega}
\end{equation}
We then use the result of $\psi_{i,j}$ back into equation \ref{psi from omega} to solve for $\psi_{i,j}$ iterately as shown in equation \ref{psi iterative}:
\begin{equation}
	\psi_{i,j}^{(k+1)} = \frac{{\Delta x}^2 {\Delta y}^2  }{2({\Delta x}^2  + {\Delta y}^2)} (\frac{\psi_{i+1,j}^{(k)} +\psi_{i-1,j}^{(k)}  }{{\Delta x}^2} + \frac{\psi_{i,j+1}^{(k)} +\psi_{i,j-1}^{(k)}  }{{\Delta y}^2 } + \omega_{i,j}).
	\label{psi iterative}
\end{equation}
Where the superscript $(k)$ means the result of the $k^{th}$ iteration of the above equation and this operation us applied to all points in $\mathcal{D}'$.
 We terminate this iterative method once the error $\sum_{(i,j) \in \mathcal{D'}} \mid \psi_{i,j}^{(k+1)} - \psi_{i,j}^{(k)} \mid$ gets suffeciently small. A similar method is applied to the polar coordinate case. 
 



\subsubsection*{Solution Stability and Accuracy}
%% Here I want to look at the stabiliy of this method

 


\subsubsection*{Lattice Boltzman Method}


{\it{In this section I give a brief introduction to the Lattice Boltzman Method and why its different from most numerical tehcniques. I introduce a 2-dimensional lattice D2Q9 and discribe the Thermal Lattice Boltzman method (TLBM) which is employed by my numerical solution to simulate convection.}}
\vspace{0.3cm}

\noindent The Lattice Boltzman Method (LBM) is a generalization of a Lattice Gas Automata (LGA), which are themselves a 
specialized Automata for simulating fluid flows. These Automata methods like common fluid simulation techniques 
discritize space and time. They directly simulate the state of particles or their distributions and evolve in time 
accoring to rules which give the desired macroscopic fluid properties as an emergent effect. This is fundamentally 
different from typical approaches which amount to directly numerically solving a set of partial differential equations.
\newline
\noindent To discritize space, we place nodes at locations $(x_i,y_j)=(i,j)$ with $i,j$ integers. Each node has attached to it a latitce, here the D2Q9 lattice shown in figure \ref{D2Q9}. The lattice defines unit vectors $\vec{e}_i$, $i \in \{ 0,1,2,3,4,5,6,7,8 \}$. In addition, each direction $e_i$ gets a weight $w_i$. In the D2Q9 lattice these are:
\begin{equation*}
w_i = \begin{cases}
          \frac{4}{9} \quad &\text{if}  \ i=0 \\
          \frac{1}{9} \quad &\text{if} \ i=1,2,3,4 \\
          \frac{1}{36} \quad &\text{if} \ i=5,6,7,8 \\
     \end{cases}.
\end{equation*}
\begin{figure}[h!]
	\centering
	\includegraphics{D2Q9Lattice.jpg}
	\caption{D2Q9 Lattice. Black notes repressent lattice points, vectors $e_0,..,e_8$ are the lattice vectors. Image sourced from \cite{khazaeli2015ghost}}
	\label{D2Q9}
\end{figure}
\noindent We wish to simulate convection. For this, we require the particle motion and the internal energy throughout the lattice. We define two distribution functions $f_{\alpha}(\vec{x}, t)$ and $g_{\alpha}(\vec{x}, t)$ denoting the particle and internal energy distributions along direction $\alpha$ at lattice points $\vec{x}$ and times $t$. The direction can be thought of as the direction of flow for particles or energy. Each timestep there are two steps to updating $f$ and $g$, a common streaming step:
\begin{equation}
	f_{\alpha}(\vec{x} + \vec{e}_\alpha, t + \Delta t) = f_{\alpha}(\vec{x}, t),
	\label{streaming step}
\end{equation}
and different collision steps:
\begin{equation}
	f_{\alpha}(\vec{x} + \vec{e}_{\alpha}, t + \Delta t) = f_{\alpha}(\vec{x}, t) + \frac{1}{\tau_f} (f^{eq}_{\alpha}(\vec{x}, t)  - f_{\alpha}(\vec{x}, t)) + F_{\alpha}
	\label{f collision step}
\end{equation}
\begin{equation}
	g_{\alpha}(\vec{x} + \vec{e}_{\alpha}, t + \Delta t) = g_{\alpha}(\vec{x}, t) + \frac{1}{\tau_g} (g^{eq}_{\alpha}(\vec{x}, t)  - g_{\alpha}(\vec{x}, t)) + G_{\alpha}.
	\label{g collision step}
\end{equation}
Here $F_{\alpha}$ and $G_{\alpha}$ are the forcing terms terms and $f^{eq}_{\alpha}$ and $g^{eq}_{\alpha}$ are equilibrium distributions. The relaxation times $\tau_f$ and $\tau_g$ are related to the macroscopic thermal diffusivity ($\kappa$) and kinematic viscousity $\nu$ by:
\begin{equation}
	\tau_g = \frac{3 \kappa}{S^2 \Delta t} + \frac{1}{2}
\end{equation}
and,
\begin{equation}
	\tau_f = \frac{3 \nu}{S^2 \Delta t} + \frac{1}{2}.
\end{equation}
The equilibrium distributions are given by the BKG approximation:
\begin{equation}
	f^{eq}_{\alpha}(\vec{x}, t)  = \rho w_{\alpha} (1 + 3 \frac{\vec{e}_{\alpha} \cdot \vec{u}}{s^2} + \frac{9}{2} \frac{(\vec{e}_{\alpha} \cdot \vec{u}  )^2}{c^4} - \frac{3}{2} \frac{\vec{u} \cdot \vec{u}}{c^2}  ),
\end{equation}
\begin{equation}
	g^{eq}_{\alpha}(\vec{x}, t)  = \epsilon \rho w_{\alpha} (1 + 3 \frac{\vec{e}_{\alpha} \cdot \vec{u}}{s^2} + \frac{9}{2} \frac{(\vec{e}_{\alpha} \cdot \vec{u}  )^2}{c^4} - \frac{3}{2} \frac{\vec{u} \cdot \vec{u}}{c^2}  ),
\end{equation}

Here $\rho$, $\epsilon$ and $\vec{u}$ are the macroscopic density, internal energy and velocity given by:
\begin{equation}
	\rho = \sum_{i=0}^{8} f_{i},
	\label{LBM rho}
\end{equation}
\begin{equation}
	\rho \vec{u} = \sum_{i=0}^{i=8} f_{i} \vec{e}_{i}
	\label{LBM u}
\end{equation}
and,
\begin{equation}
	\rho \epsilon = \sum_{i=0}^{i=8} g_{i}.
	\label{LBM ep}
\end{equation}
The forcing terms $F_i$ and $G_i$ are problem dependent. For thermal convection we only need thermal


\subsubsection*{Boundary Conditions}
%% here I want to go over the boundary conditions that I used 





\section*{Advection and Heat Conservation}
We next tested the accuracy of the advection schemes. Thermal diffusivity ($\kappa$) and thermal expansion ($\alpha$) were set to zero to avoid convection and a segment of the fluid was set to 1 unit of temperature with all other regions set to 0 units of temperature. In the first test, the streamfunction was set so that the velocity field was purely horizontal and azimuthal in the cartesian and polar cases respectively. 

\begin{figure}
	\centering
	\includegraphics{frog.png}
	\caption{Polar azimuthal advection test. Temperature (color) is plotted in polar space. Red dot indicates temperature weighted mean position within the fluid and is taken as the location of the temperature 1 zone. 
	(a) shows the initial condition, (b) shows the fluid being advected through the periodic boundry condition, (c) the analytic result for when the fluid would return to its initial state and (d) the final state of the fluid.
	Times are given above each pane.}
\end{figure}


\begin{figure}
	\centering
	\includegraphics{frog.png}
	\caption{Azimuthal advection test. Temperature (color) is plotted in polar space. }
\end{figure}

The second set of tests moved the fluid in a diagonal motion over the domain bringing it back to its origin state. 

\begin{figure}
	\centering 
	\includegraphics{frog.png}
	\caption{Cartesian diagonal advection test. Temperature (color) is plotted in polar space. Red dot indicates temperature weighted mean position within the fluid and is taken as the location of the temperature 1 zone.}
\end{figure}

I will also have a test of heat conservation. Here. This will be simple, just a few pictures of the codes running (show some nice plumes) and the energy vs time graph. It shows that neither of these conserve thermal energy.


\section*{Results}

\section*{}



\bibliographystyle{abbrv}
\bibliography{ref.bib}





\end{document}
